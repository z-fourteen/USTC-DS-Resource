\documentclass[12pt,a4paper]{article}

% ======================
% 基础宏包
% ======================
\usepackage{ctex}
\usepackage{amsmath,amssymb}
\usepackage{graphicx}
\usepackage{booktabs}
\usepackage{geometry}
\usepackage{setspace}
\usepackage{hyperref}
\usepackage{bookmark}

% ======================
% 页面设置
% ======================
\geometry{
  left=2.5cm,
  right=2.5cm,
  top=2.5cm,
  bottom=2.5cm
}
\onehalfspacing

% ======================
% 标题信息
% ======================
\title{\textbf{Lab5: Break the Barrier - Experiment Report}}
\author{学号:PB24061302 \\
姓名:赵国华 \\
院系:人工智能与数据科学学院}
\date{}

% ======================
% 正文开始
% ======================
\begin{document}
\maketitle
% ======================

\section{实验目的}
\begin{itemize}
    \item 理解strcpy函数overflow的底层原理
    \item 掌握利用缓冲区溢出进行简单攻击的方法
    \item 进一步熟悉trap vectors和LC3工具中的不同模式
\end{itemize}



\section{实验思路}
\begin{itemize}
    \item 编写合适的字符串存入RO,由20个随机字符+攻击程序
    \item 调用trap x30,触发溢出
\end{itemize}

\section{实验步骤}
\subsection{将攻击程序写入R0}
使用LEA指令将字符串地址加载到R0,字符串组成:
\begin{itemize}
    \item 利用.FILL伪指令填充20个随机字符
    \item 使用汇编语言编写跳转到x4000的攻击程序
    \item 数字x0000表示字符串解释
\end{itemize}
值得注意的点:
\begin{itemize}
    \item 加载所需的x4000地址时使用LD指令+标签模式,标签也要写到字符串内,否则会出现寻址错误
    \item 不能使用.STRINGZ伪指令,否则会在字符串末尾自动添加一个0,导致攻击程序无法正确执行
    \item 不能使用.BLKW伪指令,因为是随机取址,可能会覆盖x4000的代码
\end{itemize}
\subsection{触发溢出}
使用TRAP x30指令触发strcpy函数,导致溢出并执行攻击程序。
\section{实验结果}
成功跳转到x4000地址,输出"I make it!",实验成功。
\begin{figure}[ht]
    \centering
    \includegraphics[width=1.0\textwidth]{res1.png}
    
\end{figure}
\begin{figure}[ht]
    \centering
    \includegraphics[width=1.0\textwidth]{res2.png}
\end{figure}
\section{实验总结}
通过本次实验,深入理解了strcpy函数溢出的原理及其利用方法,掌握了缓冲区溢出攻击的基本技巧,并熟悉了trap vectors和LC3工具的不同模式操作。
发觉自己对于LEA,LD,LDI等指令的使用还不够熟练,今后需要多加练习。
\end{document}