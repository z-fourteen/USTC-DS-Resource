\documentclass[12pt,a4paper]{article}

% ======================
% 基础宏包
% ======================
\usepackage{ctex}
\usepackage{amsmath,amssymb}
\usepackage{graphicx}
\usepackage{booktabs}
\usepackage{geometry}
\usepackage{setspace}
\usepackage{hyperref}
\usepackage{bookmark}

% ======================
% 页面设置
% ======================
\geometry{
  left=2.5cm,
  right=2.5cm,
  top=2.5cm,
  bottom=2.5cm
}
\onehalfspacing

% ======================
% 标题信息
% ======================
\title{\textbf{LabA: 汇编器}}
\author{学号:PB24061302 \\ 
姓名:赵国华 \\
院系:人工智能与数据科学学院}
\date{}

% ======================
% 正文开始
% ======================
\begin{document}
\maketitle
% ======================

\section{实验目的}
\begin{itemize}
  \item 掌握汇编原理:实现双遍扫描,理解符号表与指令编码
  \item 理解指令周期:实现指令集模拟器,深入理解取值、译码、执行、访存、写回的全过程
  \item 系统集成能力:通过脚本或主程序将各个组件串连起来,实现完成的本地测试程序
\end{itemize}


\section{实验思路}
\begin{enumerate}
  \item 设计汇编器的整体架构,确定各个模块的功能和接口。
  \item 实现词法分析器,能够识别汇编语言中的各种符号和指令。
  \item 实现语法分析器,能够根据汇编语言的语法规则生成抽象语法树。
  \item 实现符号表管理模块,能够处理标签和变量的定义与引用。
  \item 实现指令编码模块,将汇编指令转换为机器码。
\end{enumerate}

根据以上分析,需要进行模块的拆分:
\begin{itemize}
  \item 输入输出模块:负责读取汇编代码文件,并将其存储为适合处理的数据结构。将生成的机器码写入输出文件。
  \item 预处理模块:负责去除注释和空白行,处理大小写不敏感的字符。
  \item 符号表模块:负责管理标签和变量的定义与引用,构建符号表。
  \item 指令集模块:负责汇编指令的格式、操作数类型和位字段布局。
  \item 错误处理模块:负责检测和报告汇编过程中出现的错误。
  \item 主控制模块:负责协调各个模块的工作流程。
\end{itemize}
\section{实验过程}
\subsection{编程语言选择与实验环境搭建}
\begin{itemize}
  \item 选择c++作为编程语言,更接近底层,便于加深对汇编器工作原理的理解。
  \item 选择VScode+WSL+CMake+Clang作为实验环境,便于跨平台开发与调试。
  \item 使用Git进行版本控制,便于代码管理与协作开发。
\end{itemize}
\subsection{汇编器设计与实现}
\subsubsection{Utils类:文件输入与输出}
\begin{itemize}
  \item 功能:处理文件的读写操作
  \item public静态成员函数:
  \begin{itemize}
    \item std::string readFile(const std::string\& filePath):以二进制模式读取文件内容并返回为字符串
    \item bool writeFile(const std::string\& filePath, const std::string\& content):将内容写入指定文件
  \end{itemize}
\end{itemize}

\subsubsection{SymbolTable类:符号表管理}
\begin{itemize}
  \item 功能:管理标签和变量的地址映射
  \item public成员函数:
  \begin{itemize}
    \item void addSymbol(const std::string\& name, int address):添加符号及其地址
    \item int getAddress(const std::string\& name) const:获取符号的地址
    \item bool contains(const std::string\& name) const:检查符号是否存在
    \item unordered\_map<std::string, int> getTable() const:获取完整的符号表
  \end{itemize}
\end{itemize}

\subsubsection{InstructionSet类:指令集定义与管理}
\begin{itemize}
  \item 功能:定义LC-3指令的格式、操作数类型和位字段布局
  \item 核心数据结构:
  \begin{itemize}
    \item BitField:描述指令中字段的位置和大小(startBit和length)
    \item OperandMode:描述特定操作数类型下的指令格式(如ADD的寄存器模式和立即数模式)
    \item InstructionInfo:包含指令名、操作码、所有支持的操作数模式
  \end{itemize}
  \item 支持的指令:ADD、AND、NOT、LD、ST、LDR、STR、LDI、STI、BR/BRN/BRZ/BRP/BRNZ/BRNP/BRZP/BRNZP、JSR、JSRR、JMP、RET、TRAP、HALT、PUTS等
  \item public成员函数:
  \begin{itemize}
    \item bool isInstruction(const std::string\& word) const:判断是否为指令
    \item bool isPseudoOp(const std::string\& word) const:判断是否为伪指令(.ORIG、.END、.FILL、.BLKW、.STRINGZ)
    \item const InstructionInfo* getInstructionInfo(const std::string\& mnemonic) const:获取指令信息
    \item const OperandMode* getOperandMode(...):根据操作数类型查找对应的模式
    \item const BitField* getBitField(const OperandMode* mode, const std::string\& fieldName) const:获取位字段信息
  \end{itemize}
\end{itemize}

\subsubsection{Assembler类:汇编器的核心}
\begin{itemize}
  \item 功能:整合预处理、第一遍扫描、第二遍扫描和代码生成
  \item private成员变量:
  \begin{itemize}
    \item std::vector<std::string> processedCode:预处理后的代码行
    \item SymbolTable symbolTable:符号表,存储标签和变量的地址映射
    \item InstructionSet instructionSet:指令集定义
    \item std::vector<std::string> machineCodes:生成的机器码
    \item int currentLineNum:用于错误报告的当前行号
    \item int origAddress:.ORIG指令指定的起始地址
  \end{itemize}
  \item private成员函数:
  \begin{itemize}
    \item void preprocess(const std::string\& sourceCode):预处理源代码
    \begin{itemize}
      \item 删除注释(分号后的内容)
      \item 删除前导和尾部空白
      \item 将所有字符转换为大写(除了.STRINGZ中的字符串)
      \item 将逗号转换为空格
    \end{itemize}
    \item void firstPass():第一遍扫描,构建符号表
    \begin{itemize}
      \item 遍历所有行,识别标签定义
      \item 计算每条指令的地址(考虑.STRINGZ和.BLKW占用的空间)
      \item 处理.STRINGZ时正确计算转义序列的实际字符数
      \item 处理.BLKW时累加块的大小
    \end{itemize}
    \item void secondPass():第二遍扫描,生成机器码
    \begin{itemize}
      \item 处理指令并生成对应的机器码
      \item 识别操作数类型(寄存器、立即数、偏移量)
      \item 进行指令编码,包括:
      \begin{itemize}
        \item 设置操作码
        \item 处理立即数标志位(ADD/AND的位5)
        \item 处理分支条件位(BR指令的位9-11)
        \item 处理特殊标志位(JSR的位11)
        \item 处理陷阱向量(TRAP相关指令)
        \item 处理RET为特殊的JMP R7
      \end{itemize}
      \item 处理伪指令(.ORIG、.FILL、.BLKW、.STRINGZ、.END)
      \item 实现转义序列处理(\textbackslash n、\textbackslash t等)
      \item 完善的错误检测和异常抛出
    \end{itemize}
    \item std::string generateMachineCodeOutput():生成输出格式的机器码
  \end{itemize}
  \item public成员函数:
  \begin{itemize}
    \item void assemble(std::string inputPath, std::string outputPath):主汇编接口
    \begin{itemize}
      \item 读取输入文件
      \item 调用preprocess、firstPass、secondPass
      \item 将机器码写入输出文件
      \item 异常处理和错误报告
    \end{itemize}
  \end{itemize}
\end{itemize}

\subsubsection{AssemblerException类:异常处理}
\begin{itemize}
  \item 功能:统一的异常处理机制,包含8种错误代码:
  \begin{itemize}
    \item FILE\_READ\_ERROR (1):文件读取失败
    \item FILE\_WRITE\_ERROR (2):文件写入失败
    \item SYNTAX\_ERROR (3):语法错误(如缺少操作数)
    \item SEMANTIC\_ERROR (4):语义错误(如未定义的标签)
    \item INVALID\_OPERAND (5):无效操作数(如寄存器范围超出)
    \item INVALID\_INSTRUCTION (6):未知指令
    \item SYMBOL\_ERROR (7):符号表错误(如重复定义)
    \item VALUE\_OUT\_OF\_RANGE (8):值超出范围(如立即数过大)
  \end{itemize}
  \item 特点:包含行号信息,便于定位错误位置
\end{itemize}

\section{实验结果}

\subsection{实现的功能特性}
\begin{itemize}
  \item ✅ 完整的LC-3汇编器实现,支持15+条指令
  \item ✅ 双遍扫描架构:第一遍构建符号表,第二遍生成机器码
  \item ✅ 多模式指令支持:如ADD/AND的寄存器和立即数模式
  \item ✅ 完善的伪指令处理:.ORIG、.END、.FILL、.BLKW、.STRINGZ
  \item ✅ 转义序列处理:支持$\backslash$n、$\backslash$t等特殊字符
  \item ✅ 8种详细的错误代码和行号定位
  \item ✅ 特殊指令处理:BR/BRN等分支指令的条件位、RET作为JMP R7的转换
  \item ✅ 大小写敏感处理:保留.STRINGZ中字符串的原始大小写
\end{itemize}

\subsection{关键算法与实现细节}

\subsubsection{预处理阶段的大小写处理}
预处理需要将指令和操作数转换为大写以便统一处理,但需要保留.STRINGZ字符串的原始大小写。实现方式:
\begin{itemize}
  \item 检测行中是否包含.STRINGZ
  \item 若有,将.STRINGZ之前的部分转换为大写,之后的部分保留原样
  \item 若无,全行转换为大写
\end{itemize}

\subsubsection{第一遍扫描的地址计算}
不同伪指令占用的地址空间不同:
\begin{itemize}
  \item 普通指令:1个地址单位
  \item .STRINGZ:字符数+1(null终止符),需考虑转义序列
  \item .BLKW:指定的块大小
  \item .FILL:1个地址单位
\end{itemize}

对于.STRINGZ的长度计算,需要逐个扫描字符,识别转义序列(如$\backslash$n、$\backslash$t等),每个转义序列只计为1个字符。

\subsubsection{第二遍扫描的机器码生成}
关键要点:
\begin{itemize}
  \item 操作数识别:根据首字符判断类型(R开头为寄存器,\#或X开头为立即数,其他为标签)
  \item 立即数标志位:ADD和AND指令若第三个操作数为立即数,则设置位5为1
  \item 分支条件位:BR指令根据指令名确定位9-11(NZP位):
  \begin{itemize}
    \item BRn:100、BRz:010、BRp:001、BRNZ:110等
    \item BR(无条件):111
  \end{itemize}
  \item 陷阱向量:TRAP系列指令设置对应的向量号(0x20-0x25)
  \item RET特殊处理:转换为固定的机器码0x1C00(JMP R7)
  \item 转义序列处理:.STRINGZ中的$\backslash$n等转换为对应的ASCII码
\end{itemize}

\subsection{测试结果}
使用提供的compare\_results.py脚本对比输出与标准答案:
\begin{itemize}
  \item 测试用例覆盖:fibonacci、加法、与、非、分支、跳转、子程序、栈操作、字符串处理等
  \item 验证内容:指令编码、地址计算、符号表、特殊指令处理
  \item 测试工具:Python脚本自动化对比,显示差异详情和十六进制差值
\end{itemize}

\section{实验总结}

\subsection{关键问题与解决方案}

\subsubsection{问题1:大小写敏感性的处理}
汇编指令应该大小写不敏感,但字符串内容需要保留原始大小写。
\begin{itemize}
  \item 问题:直接转换整行为大写会破坏字符串的大小写
  \item 解决:在preprocess函数中检测.STRINGZ,分别处理指令部分和字符串部分
\end{itemize}

\subsubsection{问题2:立即数标志位的编码}
ADD和AND指令支持两种模式,但立即数模式需要在位5设置标志位。
\begin{itemize}
  \item 问题:简单的位字段定义无法表示固定的标志位
  \item 解决:在secondPass中,根据操作数类型检测,若为立即数则在位5设置1
\end{itemize}

\subsubsection{问题3:字符串长度的精确计算}
.STRINGZ的字符数需要考虑转义序列,而不是简单地用字符串长度减去引号。
\begin{itemize}
  \item 问题:直接计数会把转义序列(如$\backslash$n)算为2个字符
  \item 解决:逐个字符扫描,识别转义模式并跳过反斜杠
\end{itemize}

\subsubsection{问题4:RET指令的特殊处理}
RET(返回)指令实际是JMP R7的特殊情况。
\begin{itemize}
  \item 问题:RET没有操作数,无法通过通用的指令编码路径处理
  \item 解决:在secondPass中特殊检测RET指令,直接设置机器码为0x1C00
\end{itemize}

\subsubsection{问题5:分支偏移量的范围检查}
分支指令使用相对寻址,偏移量必须在指定范围内。
\begin{itemize}
  \item 问题:无效的偏移量会导致寻址错误
  \item 解决:计算pcOffset后检查范围$[-2^{n-1}, 2^{n-1}-1]$,超出范围抛出异常
\end{itemize}

\subsection{设计亮点}

\begin{itemize}
  \item \textbf{模块化设计}:分离关注点,Utils处理I/O、SymbolTable管理符号、InstructionSet定义指令格式、Assembler协调整个流程
  \item \textbf{异常处理}:8种错误代码覆盖各类问题,包含行号信息便于定位和调试
  \item \textbf{多模式支持}:用OperandMode数组实现多种操作数模式,灵活应对不同指令格式
  \item \textbf{位字段精确描述}:BitField结构明确定义每个字段的位置和大小,便于自动化编码
  \item \textbf{代码一致性}:注释全部中文,便于代码审阅和维护
\end{itemize}

\subsection{学习收获与反思}

通过本实验,深入理解了:
\begin{itemize}
  \item LC-3汇编语言的完整语法和语义
  \item 汇编器的设计原理和实现流程(预处理、符号表构建、代码生成)
  \item 机器码的二进制编码,位字段的精确定位
  \item 异常处理和错误恢复的重要性
  \item 模块化设计在复杂系统中的价值
  \item 测试驱动开发的必要性(自动化对比工具快速反馈)
\end{itemize}

\end{document}